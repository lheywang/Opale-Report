\subsection{The rotational equation to make our model with 6DoF}
We now have our equations to compute the position and its derivatives over
time. The thing is those equations are only for a rigid body with only 3 DoF
and without any ailerons added to them. To be able to have our 6 DoF and to be
able to input our ailerons into this simulation we need to refer to Euler's
equation of rotational dynamics as well as the standard aerodynamic lift
equations:

For a rigid body\footnote{rigid body is symmetrical and the axes are aligned
    with the frame} with a diagonal inertia tensor:
\begin{gather*}
    \begin{cases}
        r_x = I_x \dot{\omega}_x - (I_y - I_z)\omega_y \omega_z \\
        r_y = I_y \dot{\omega}_y - (I_z - I_x)\omega_z \omega_x \\
        r_z = I_z \dot{\omega}_z - (I_x - I_y)\omega_x \omega_y
    \end{cases}
\end{gather*}

With $r$ the net torque, $I_\omega$ the angulor momentum, $I$ the inertia
tensor and $\omega$ is the angular velocity.

We can assume that the rocket is symmetrical so $I_x=I_y$ and we could
calculate the $I_z$ but we would see that if the rocket is long but slim it
tends to be small. We consider as well that the pitch and yaw are coupled but
the roll isn't coupled with the rest.

We now find thoses equations :
\begin{gather*}
    \begin{cases}
        r_x = I_x \dot{\omega}_x - (I_x - I_z)\omega_y \omega_z \\
        r_y = I_x \dot{\omega}_y - (I_z - I_x)\omega_z \omega_x \\
        r_z = I_z \dot{\omega}_z
    \end{cases}
\end{gather*}

We can integrate those equations into the simulation like this:

\begin{align*}
    \dot{\omega}_x & = \frac{1}{I_x}(r_x + (I_x - I_z)\omega_y \omega_z) \\
    \dot{\omega}_y & = \frac{1}{I_x}(r_y + (I_z - I_x)\omega_z \omega_x) \\
    \dot{\omega}_z & = \frac{I_z}r_z                                     \\
\end{align*}

And we can integrate them over time to get orientation rates like this :

\begin{gather*}
    \dot{\omega}_i(t + dt) = \omega_i(t) + \dot{\omega}_i dt
\end{gather*}

Because we use Euler's equations, we encounter a problem known as the gimbal
lock. This happens as two rotation axes become aligned. Because we are in a 3D
space instead of a 4D space like for quaternions, we can't differentiate
between those two axes and so we lose a DoF as long as those axes are merged.

We now need to talk about a problem to determine our moments of inertia for our rocket.
These moments are governed by those equations:

\begin{align*}
    I_x = I_y & = \frac{1}{12} \times m(r^3 + L^2)I_z \\
              & = \frac{1}{2} \times mr^2             \\
\end{align*}
with $m$ the weight, $r$ the radius and $L$ the length.
Our weight goes from around 1690g to 1350g so a loss of about 20\% the total initial
weight which isn't negligeable. To simplify our equation with a linear interpolation
to these equations which are much simpler to implement into our simulation:

\begin{align*}
    I_x(t) = I_y(t) & = I_{x, dry} + \frac{1}{12}m_{\text{fuel}}(t) \times L^2_{\text{motor}} + m_{\text{fuel}}(t)dt^2 \times I_z(t) \\
                    & = I_{z, dry} + \frac{1}{8} \times m_{\text{fuel}}(t) \times D^2                                                \\
\end{align*}

With $m$ fuel the remaining fuel at the instant t, $D$ the diameter of the rocket and $L$
motor the length of the motor. Now that we have our rotational d Iynamic, we can calculate
Euler Angle Kinematics which is used to convert our angular velocities in our body frame
referential to Euler angle rates. To do this we use this transformation:

\begin{gather*}
    \begin{bmatrix}
        \dot{\Omega} \\
        \dot{\Theta} \\
        \dot{\Psi}   \\
    \end{bmatrix}
    =
    \begin{bmatrix}
        1 & \sin{\Phi} \times \tan{\Theta}  & \cos{\Phi} \times \tan{\Theta}  \\
        0 & \cos{\Phi}                      & -\sin{\Phi}                     \\
        0 & \frac{\sin{\Phi}}{\cos{\Theta}} & \frac{\cos{\Phi}}{\cos{\Theta}} \\
    \end{bmatrix}
    \times
    \begin{bmatrix}
        \omega_x \\
        \omega_y \\
        \omega_z \\
    \end{bmatrix}
\end{gather*}