\section{The control theory and integration into simulation}

We will now talk about the equations and the control theory behind Opale to be
able to simulate it later and to better comprehend what physical phenomenon
will impact the flight of our rocket to be more precise in the conception of
the body. One thing to be noted is that we will use Euler’s equation to
comprehend those

\subsection{Simplified Dynamics of the Rocket with 3 DoF}

This rocket can be modeled as a rigid body with 3 degrees of freedom (DoF), but
the control can be focused primarily on pitch and yaw. To follow everything
below we will now see about a 3D case of our rocket this part has been heavily
inspired by the document “Le vol de la Fusée, Stabilité et Trajectographie” by
the CNES:

\begin{figure}
    \centering
    \resizebox{\SchematicWidth}{!}{%
    \begin{tikzpicture}
        % adding grid
        \draw [->, line width=0.5pt] (0,-0.5) -- (0,10) node [left] {Z};
        \draw [->, line width=0.5pt] (-0.5,0) -- (10,0) node [below] {X};

        % Adding arrows
        % Horizontals
        \draw[->, line width=2pt] (5.8,0) -- (4,0) node[midway, below]{\(-R\cos{\theta}\)};
        \draw[->, line width=2pt] (6.2,0) -- (8,0) node[midway,below] {\(Pcos{\theta}\)};
        % Vertical
        \draw[->, line width=2pt] (0,5.8) -- (0,4) node[left] {\(-R\sin{\theta}\)};
        \draw[->, line width=2pt] (0,5.8) -- (0,4.4) node[left] {\(-M_g\)};
        \draw[->, line width=2pt] (0,6.2) -- (0,8) node[left] {\(P\sin{\theta}\)};

        % small lines that make the graphe readable
        % Horizontal lines
        \draw [dotted, line width=0.5pt] (0.5,4) -- (4,4);
        \draw [dotted, line width=0.5pt] (0.5,6) -- (6,6);
        \draw [dotted, line width=0.5pt] (0.5,4.4) -- (6, 4.4);
        \draw [dotted, line width=0.5pt] (0.5,8) -- (8,8);
        % Vertical lines
        \draw [dotted, line width=0.5pt] (4,0.5) -- (4,4);
        \draw [dotted, line width=0.5pt] (5.6,0.5) -- (5.6,5.6);
        \draw [dotted, line width=0.5pt] (6,0.5) -- (6,6);
        \draw [dotted, line width=0.5pt] (8,0.5) -- (8,8);

        % Adding the rocket
        \path [draw=black, fill=yellow, line width=0.1mm]
            (4.625,3.875) -- ++(0.6,0.6) -- ++(0,0.5) -- ++(1.1,1.1) --
            ++ (0.5,0.75) -- ++(-0.75,-0.5) -- ++(-1.1,-1.1) --
            ++ (-0.5,0) -- ++(-0.6,-0.6) -- ++(0.5,0) -- ++(0.25,-0.25)
            -- ++(0,-0.5);

        % Adding forces vector
        \draw [->, line width=1pt] (5.6,5.6) node [above left] {CP}-- (4,4) node [below left] {\(\overrightarrow{R}\)};
        \draw [->, line width=1pt] (6,6) -- (8,8) node [above] {\(\overrightarrow{P}\)};
        \draw [->, line width=1pt] (6,6) node [below right] {CG} -- (6,4.4) node [right] {\(\overrightarrow{M_g}\)};

        % Adding angle mark
        \draw (1,1) coordinate(B) -- (2,1) coordinate(A);
        \draw (1,1) -- (2,2) coordinate(C);
        \pic [draw, ->, "$\theta$", angle eccentricity=1.5] {angle = A--B--C};
        
    \end{tikzpicture}
    }
    \caption{Forces applied to the rocket}\label{img:rocket_forces}
\end{figure}

\FloatBarrier

If we look a bit into the forces on the rocket in flight we can get the figure
just above and we can take out those 2D equations rather easily:

\begin{align*}
    P_x      & = P \times \cos({\theta})   \\
    R_x      & = - R \times \cos({\theta}) \\
    P_z      & = P \times \sin({\theta})   \\
    Weight_Z & = -M \times g               \\
    R_z      & = -R \times \sin ({theta})  \\
\end{align*}
\FloatBarrier

Now that we have our basic equations, we can determine our equations based on
time and in 3D :

\begin{align*}
    t_i  & = t_{i-1} + dt \text{: Compute the actual time}                                                                        \\
    m_i  & = m_{i-1} - dm \text{: compute the mass of the roceket as the motor burns fuel}                                        \\
    P_i  & = \text{mean thrust between $t_i$ and $t_{i-1}$}                                                                       \\
    Fr_i & = (P_i - \frac{1}{2}\rho(Z_{i-1}SCx(V_{i-1})V^2_{i-1}) \cos({\text{pitch}_{i-1}})) \text{: Horizontales sum of forces} \\
    Fz_i & = (P_i - \frac{1}{2}\rho(Z_{i-1}SCx(V_{i-1})V^2_{i-1}) \sin({\text{pitch}_{i-1}})) \text{: Vertical sum of forces}     \\
    Gr_i & = \frac{Fr_i}{m_i} \text{: Horizontale acceleration}                                                                   \\
    Gz_i & = \frac{Fz_i}{m_i} \text{: Vertical acceleration}                                                                      \\
    Vr_i & = Vr_{i-1} + Gr_i dt \text{: Horizontale velocity}                                                                     \\
    Vz_i & = Vz_{i-1} + Gz_i dt \text{: Vertical velocity}                                                                        \\
    V_i  & = \sqrt{Vr_i^2 + Vz_i^2} \text{: Total velocity}                                                                       \\
    X_i  & = X_{i-1} + (Vr_i dt + Gr_i \frac{dt^2}{2}) \cos({yaw})\text{: X position}                                             \\
    Y_i  & = Y_{i-1} + (Vr_i dt + Gr_i \frac{dt^2}{2}) \sin({yaw}) \text{: Y position}                                            \\
    Z_i  & = Z_{i-1} + Vzi dt + Gz_i \frac{dt^2}{2} \text{: Z position}                                                           \\
\end{align*}

And, if the rocket is stil on the launch ramp :

\begin{align*}
    Fr_i & = Fr_i - m_i g cos({pitch_{i-1}}) sin({pitch_{i-1}}) \\
    Fz_i & = Fz_i - m_i g cos({pitch_{i-1}}) cos({pitch_{i-1}}) \\
\end{align*}

Those equations can be exported into simulation software like Simulink
following this flow chart

\begin{figure}[!ht]
    \centering
    \resizebox{0.5\textwidth}{!}{%
        \begin{circuitikz}
            \tikzstyle{every node}=[font=\large]
            \draw [ fill={rgb,255:red,168; green,255; blue,168} ] (12.5,30.75) rectangle  node {\large $\text{Start of calculations :} M_0, V_0, Z_0, \theta_0$} (25,28.25);
            \draw [ fill={rgb,255:red,125; green,190; blue,255} ] (12.5,25.75) rectangle  node {\large $\text{Calculate} M_i$} (25,23.25);
            \draw [ fill={rgb,255:red,125; green,190; blue,255} ] (5,20.75) rectangle  node {\large $\text{Calculate} \gamma X_i (\text{Depends} n R_{i-1} and cos \theta_{i-1})$} (17.5,18.25);
            \draw [ fill={rgb,255:red,125; green,190; blue,255} ] (20,20.75) rectangle  node {\large $\text{Calculate} \gamma Y_i (\text{Depends} n R_{i-1} and sin \theta_{i-1})$} (32.5,18.25);
            \draw [ fill={rgb,255:red,125; green,190; blue,255} ] (5,15.75) rectangle  node {\large $\text{Calculate} V\gamma_i (\text{Depends on} Vx and \gamma_i)$} (17.5,13.25);
            \draw [ fill={rgb,255:red,125; green,190; blue,255} ] (20,15.75) rectangle  node {\large $\text{Calculate} Vz_i (\text{Depends on} Vz_{i-1} and \gamma_i )$} (32.5,13.25);
            \draw [ fill={rgb,255:red,125; green,190; blue,255} ] (5,10.75) rectangle  node {\large $\text{Calculate} the x_i (\text{Depends on} X_{i-1} and \gamma X_i)$} (17.5,8.25);
            \draw [ fill={rgb,255:red,125; green,190; blue,255} ] (20,10.75) rectangle  node {\large $\text{Calculate} Z_i (\text{Depends on} Z_{i-1} and \gamma Z_i)$} (32.5,8.25);
            \draw [ fill={rgb,255:red,125; green,190; blue,255} ] (12.5,5.75) rectangle  node {\large $\text{Calculate} V_i$} (25,3.25);
            \draw [ fill={rgb,255:red,125; green,190; blue,255} ] (12.5,0.75) rectangle  node {\large $\text{Calculate} R_i$} (25,-1.75);
            \draw [ fill={rgb,255:red,125; green,190; blue,255} ] (12.5,-4.25) rectangle  node {\large $\text{Calculate} \theta_i$} (25,-6.75);
            \draw [ fill={rgb,255:red,251; green,223; blue,249} ] (12.5,-9.25) rectangle  node {\large Does the rocket has landed ?} (25,-11.75);
            \draw [ fill={rgb,255:red,255; green,128; blue,128} ] (12.5,-14.25) rectangle  node {\large End of calculations} (25,-16.75);
            \draw [->, >=Stealth] (18.75,28.25) -- (18.75,25.75);
            \draw [->, >=Stealth] (11.25,22) -- (11.25,20.75);
            \draw [->, >=Stealth] (26.25,22) -- (26.25,20.75);
            \draw [->, >=Stealth] (11.25,18.25) -- (11.25,15.75);
            \draw [->, >=Stealth] (26.25,18.25) -- (26.25,15.75);
            \draw [->, >=Stealth] (11.25,13.25) -- (11.25,10.75);
            \draw [->, >=Stealth] (26.25,13.25) -- (26.25,10.75);
            \draw [->, >=Stealth] (18.75,7) -- (18.75,5.75);
            \draw [->, >=Stealth] (18.75,3.25) -- (18.75,0.75);
            \draw [->, >=Stealth] (18.75,-1.75) -- (18.75,-4.25);
            \draw [short] (11.25,22) -- (26.25,22);
            \draw [short] (18.75,23.25) -- (18.75,22);
            \draw [short] (11.25,8.25) -- (11.25,7);
            \draw [short] (11.25,7) -- (26.25,7);
            \draw [short] (26.25,8.25) -- (26.25,7);
            \draw [->, >=Stealth] (18.75,-6.75) -- (18.75,-9.25);
            \draw [->, >=Stealth] (18.75,-11.75) -- (18.75,-14.25)node[pos=0.5, fill=white]{YES};
            \draw [short] (12.5,-10.5) -- (2.5,-10.5)node[pos=0.5, fill=white]{NO};
            \draw [short] (2.5,-10.5) -- (2.5,24.5);
            \draw [->, >=Stealth] (2.5,24.5) -- (12.5,24.5);
        \end{circuitikz}
    }%
    \caption{Algorithm used to simulate the rocket trajectory.}
    \label{fig:my_label}
\end{figure}