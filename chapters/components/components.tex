When starting the project, one of the first points was to choose components.
But, why this one ? And not the other ? Let's dig into the hardware choices we
made here !

\section{Servo engines}
The first component we need to choose were the servo engines. These are
responsible of the position of the wings outside of the rocket. Our criterions
were :
\begin{itemize}
    \item   Minimal error on the control position, ideally with a position feedback.
    \item   Fast enough response.
    \item   Strong enough to resist to the air pressure.
    \item   Small enough to be placed inside of the rocket.
\end{itemize}

We considered the servos listed on the \ref{tab:servo_list} table.

\input{\Tables/list_servo.tex}

Our selection end up on the KPower servo engine, because of their availabily on
well known retailers and specs that are enough for our needs.

\section{MCUs}
The second component that required a bit selection was the MCU, the
microcontroller that will control everything. We didn't needed a ton of
computing power, but we needed some advanced peripherals, and eventually an 2.4
GHz radio interface for a remote Bluetooth control.

We considered these MCUs, listed on the \ref{tab:mcu_list} table

\begin{table}[!hbt]
    \centering
    \begin{tabular}{| c || c | c |}
        \hline
        Reference & Manufacturer         & Core specifications                              \\
        \hline
        \hline
        nRF5340   & Nordic Semiconductor & ARM Cortex M33 @128 MHz + ARM Cortex M33 @64 MHz \\
        STM32H5   & STMicroElectronics   & ARM Cortex M33 @250 MHz                          \\
        STM32F4   & STMicroElectronics   & ARM Cortex M4 @84 MHz                            \\
        ATSAMD51  & Microchip            & ARM Cortex M4F @120 MHz                          \\
        \hline
    \end{tabular}
    \caption{List of considered MCUs for the project}
    \label{tab:mcu_list}
\end{table}
\FloatBarrier

Our choice end up on the nRF5340 IC, because it has enough computing power for
our need (and it has two cores !) and some advanced peripherals interconnects
(such as PPI, which enable us to connect peripherals between them to create
logic conditions !). But there where another fact : It has an excellent
software support and documentation, which are going to help us a lot. That was
mainly this point that make us select this chip, compared to the other options
which where less documented.

\section{Sensors selection}
The sensors are also an important part of the rocket, because they're the entry
of informations. They thus need to be precise enough, but, more importantly, do
not derive excessively.

And, since these sensors are mostly accelerometers, the cost for these can
easily become important. For this selection, we did had a choice between two
options :

\subsection{A Pair of accelerometers}
A single accelerometer isn't an expensive device, so it's perfectly possible to
use two or more of them on a single board.

And, with some software treatement we can get the position, speed and rotations
of the rocket with only two of them, by derivating the acceleration.

But, this first solution has a major flaw : It will drift over time, leading to
imprecise calculations, and thus : An error. However, this solution may be
viable for slowest devices.

\subsection{An integrated IMU}
To counter this aspect, we selected a more complex device, that integrate
multiple sensors into one. It enable the device to compensate the drift of a
device with another, but there is a cost : This require a complex software.

Hopefully, this software is already integrated into the chip, leaving us with
already treated data.

That's why we choosed the BNO055, an IMU rather than a pair of accelerometers.
Nonetheless, we deciced to implement both options on the board, to let us some
freedom about or solutions, and to tests different possibilites.