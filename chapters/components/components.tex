When starting the project, one of the first points was to choose components. But,
why this one ? And not the other ? Let's dig into the hardware choices we made here !

\section{Servo engines}
The first component we need to choose were the servo engines. Theses are responsible
of the position of the wings outside of the rocket.
Our criterions were :
\begin{itemize}
    \item   Minimal error on the control position, ideally with a position feedback.
    \item   Fast enough response.
    \item   Strong enough to resist to the air pressure.
    \item   Small enough to be placed inside of the rocket.
\end{itemize}

We considered the servos listed on the \ref{tab:servo_list} table.

\begin{table}[!hbt]
    \centering
    \begin{tabular}{| c || c | c | c |}
    \hline
    Reference &  Supply voltage & Position feedback & Dimensions (L x W x H)\\
    \hline
    \hline
    LEX-SERVO02 & 4.8 \si{\volt} - 6 \si{\volt} & Yes & 22 \si{\milli\meter} x 12.5 \si{\milli\meter} x 27.3 \si{\milli\meter} \\
    TinyMG 65122 & 4.8 \si{\volt} - 6 \si{\volt} & No & 30 \si{\milli\meter} x 15 \si{\milli\meter} x 30 \si{\milli\meter} \\
    KPower 9g MM090 & 4.8 \si{\volt} - 6 \si{\volt} & Yes & 22.4 \si{\milli\meter} x 12.5 \si{\milli\meter} x 22.8 \si{\milli\meter} \\
    FEETECH FT3325M & 4.8 \si{\volt} - 6 \si{\volt} & No & 30 \si{\milli\meter} x 10 \si{\milli\meter} x 35.5 \si{\milli\meter} \\
    FEETECH FT1117M & 4.8 \si{\volt} - 6 \si{\volt} & Yes & 22 \si{\milli\meter} x 12.5 \si{\milli\meter} x 27.3 \si{\milli\meter} \\
    Hitec HS85BB & 4.8 \si{\volt} - 6 \si{\volt} & No & 28 \si{\milli\meter} x 13 \si{\milli\meter} x 30 \si{\milli\meter} \\
    Datan S1123 & 4.8 \si{\volt} - 6 \si{\volt} & Yes & 32.7 \si{\milli\meter} x 15.7 \si{\milli\meter} x 31.7 \si{\milli\meter} \\
    \hline
    \end{tabular}
    \caption{List of the possibles servo engines}
    \label{tab:servo_list}
\end{table}

Our selection end up on the KPower servo engine, because of their availabily on well known
retailers and specs that are enough for our needs.

\section{MCUs}
The second component that required a bit selection was the MCU, the microcontroller that
will control everything. We didn't needed a ton of computing power, but we needed some
advanced peripherals, and eventually an 2.4 GHz radio interface for a remote Bluetooth
control.

We considered theses MCUs, listed on the \ref{tab:mcu_list} table

\begin{table}[!hbt]
    \centering
    \begin{tabular}{| c || c | c | c |}
    \hline
    Reference &  Supply voltage & Position feedback & Dimensions (L x W x H)\\
    \hline
    \hline
    LEX-SERVO02 & 4.8 \si{\volt} - 6 \si{\volt} & Yes & 22 \si{\milli\meter} x 12.5 \si{\milli\meter} x 27.3 \si{\milli\meter} \\
    TinyMG 65122 & 4.8 \si{\volt} - 6 \si{\volt} & No & 30 \si{\milli\meter} x 15 \si{\milli\meter} x 30 \si{\milli\meter} \\
    KPower 9g MM090 & 4.8 \si{\volt} - 6 \si{\volt} & Yes & 22.4 \si{\milli\meter} x 12.5 \si{\milli\meter} x 22.8 \si{\milli\meter} \\
    FEETECH FT3325M & 4.8 \si{\volt} - 6 \si{\volt} & No & 30 \si{\milli\meter} x 10 \si{\milli\meter} x 35.5 \si{\milli\meter} \\
    FEETECH FT1117M & 4.8 \si{\volt} - 6 \si{\volt} & Yes & 22 \si{\milli\meter} x 12.5 \si{\milli\meter} x 27.3 \si{\milli\meter} \\
    Hitec HS85BB & 4.8 \si{\volt} - 6 \si{\volt} & No & 28 \si{\milli\meter} x 13 \si{\milli\meter} x 30 \si{\milli\meter} \\
    Datan S1123 & 4.8 \si{\volt} - 6 \si{\volt} & Yes & 32.7 \si{\milli\meter} x 15.7 \si{\milli\meter} x 31.7 \si{\milli\meter} \\
    \hline
    \end{tabular}
    \caption{List of the possibles servo engines}
    \label{tab:servo_list}
\end{table}

Our choice end up on the nRF5340 IC, because it has enough computing power for our need
(and it has two cores !) and some advanced peripherals interconnects (such as PPI, which
enable us to connect peripherals between them to create logic conditions !).
But there where another fact : It has an excellent software support and documentation,
which are going to help us a lot. That was mainly this point that make us select this chip,
compared to the other options which where less documented.

\section{Sensors selection}
The sensors are also an important part of the rocket, because they're the entry of informations.
They thus need to be precise enough, but, more importantly, do not derive excessively.

And, since theses sensors are mostly accelerometers, the cost for theses can easily become
important. Thus, we didn't get a lot of choice, and we really got a single option ! The BNO055.

We assisted it with a pair of simpler accelerometers, mainly for testing purposes rather and
precision.