\section{Soldering}\label{sec:assembly}
For the last part of the electronic chapter, we're going to talk about
assembly of the board.

The board use complex components, where the packages are BGA, aQFN...
All of the pitch are in the $0.5 \si{\mu\meter}$ range, thus, it's
evident that it won't be possible to solder it with an iron.

To solder theses components, we used a technic based on industrial
processes, which consists of different steps :

\begin{itemize}
    \item   Apply some solder pastes on the pads.
    \item   Place the ICs and passives on their spots.
    \item   Heat the whole board, to make the solder paste melt.
    \item   Let the board cool down.
\end{itemize}

Solder paste is a mix of tin and flux. It's used to create precision
soldering, since it's much easier to get the rigth volume of tin on a
point.

\subsection{Solder paste application}
First, we need to apply solder pastes. The smallest pads are round,
$200 \si{\mu\meter}$ in diameter, and $500 \si{\mu\meter}$ far from the
other.

It's then impossible to place solder paste by hand on this pads.

That's why we used a stencil, that we place over the board, and fix in
place with different tools. Then, we apply some paste on this stencil, and
we spread it on the board with something rigid enough.

This will fill the stencil holes with the right volume of paste, and, if
correctly placed, right over the pads. Once finished, we remove the stencil,
and there shall be right enough paste, we it needs. \footnote{
    Since the paste is about half flux, getting a bit of paste where it shouldn't
    isn't generally an issue. And, with surface tension of melted tin, it will
    generally flow to the contact without issues.
}

\subsection{Placing the SMT}
Once we're satisfied with the paste application, we can start to place components.

This step can be quite long, since it require a lot of application, and concentration.
Hopefully, there is some tools to make it easier, such as Altium assembly assistant,
which will prompt us which reference goes where, and in which orientation.

This make the placement much faster and right.

Once all components are placed, we inspect them. They need to be precisely placed
for all of them\footnote{
    When the board is hot enough, again, the surface tensions of the melted tin will
    tend to place the IC by themselves. Imprecisions can be corrected here, for a
    maximum of half the distance between two pads.
}. In our board, there is 286 of them !

\subsection{Heating}
For the final part, we need to heat the board. There is multiple solutions here,
we can do it on a specific furnace on the FabLab, or, with an hotplate at home.

We tried the second solution for the first board. This plate is going to heat to
$250 \si{\degree}$, heating the whole board in the same time. All of the paste will
melt in the "same" time, and thus all of the solder will be done in one time.

Once finished, we remove the board carrefully of the heating plate, and wait for it
to cool.

All of the process is in photo right below :

\begin{figure}[!hbt]
    \centering
    \begin{minipage}[c]{\SmallSchematicWidth}
        \centering
        \includegraphics[width=\textwidth]{\Images/assembly/assembly_STENCIL.eps}
        \caption*{Stencil placement}
    \end{minipage}%
    \hfill%
    \begin{minipage}[c]{\SmallSchematicWidth}
        \centering
        \includegraphics[width=\textwidth]{\Images/assembly/assembly_PASTE.eps}
        \caption*{Solder paste applied}
    \end{minipage}%
    \hfill%
    \begin{minipage}[c]{\SmallSchematicWidth}
        \centering
        \includegraphics[width=\textwidth]{\Images/assembly/assembly_SMT.eps}
        \caption*{PCB with the SMT placed}
    \end{minipage}%
    \hfill%
    \begin{minipage}[c]{\SmallSchematicWidth}
        \centering
        \includegraphics[width=\textwidth]{\Images/assembly/assembly_HOTPLATE.eps}
        \caption*{PCB on the heating surface}
    \end{minipage}
    \label{img:assembly}
    \caption{assembly process}
\end{figure}
\FloatBarrier

After that whole process, we need to add the few trough hole components, manually.
This is quite fast, since there is not a lot of them.