\section{Testing and defaults}\label{sec:defaults}
Once the board is finished, we need to test it. And, in our case, the $3.3 \si{\volt}$
and the GND were in short-circuit. Most of the nets where fine.

To search for the location of the default, we used an electronic magnifying glass.
We found theses defaults :

\begin{figure}[!hbt]
    \centering
    \begin{minipage}[c]{0.32\textwidth}
        \centering
        \includegraphics[width=\textwidth]{\Images/assembly/default_SOLDER.eps}
        \caption*{Too much solder}
    \end{minipage}%
    \hfill%
    \begin{minipage}[c]{0.32\textwidth}
        \centering
        \includegraphics[width=\textwidth]{\Images/assembly/default_SOIC.eps}
        \caption*{Lack of heat n1}
    \end{minipage}%
    \hfill%
    \begin{minipage}[c]{0.32\textwidth}
        \centering
        \includegraphics[width=\textwidth]{\Images/assembly/default_CAP.eps}
        \caption*{Lack of heat n2}
    \end{minipage}%
    \label{img:defaults}
    \caption{assembly defaults}
\end{figure}
\FloatBarrier

\subsection{Hand soldering default}
The first default came from the use, when soldering with hand some of the
capacitor on the back side.

At first, the solder was looking quite good, but, under magnification, we
found that some contacts where touching between them.

This kind of issues are easily corrected by removing the excess solder,
with, for example a solder wick.

\subsection{Heating defaults}
The two last defaults are a bit harder to patch. They come from a lack
of heat on the board, which didn't make fully melted the solder paste.
Thus, there is some tin balls on multiple places of the PCB.

This is probably the source of our short circuits, because the nets that
are in short-circuits are always side by side under the microcontroller.

This kind of defaults can be patched by reheating the board, with enough
flux.

\subsection{Defaults removal}
When reheating the board, the defaults didn't disapeared, leaving us with a
defective board.

We then soldered two anothers boards, with a slightly different technique to
apply solder paste. This method gave us greater results, and we end up with
two working boards.
