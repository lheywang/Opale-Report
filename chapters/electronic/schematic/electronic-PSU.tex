\section{Power supplies}
Now, let's look a bit deeper on the power supplies, and how they're agenced.
\subsection{Power tree}
To represent the whole power supply organization, we drew a power tree, a schematic that
represent the power supplies.

\begin{figure}[!ht]
    \centering
    \resizebox{\SchematicWidth}{!}{%
        \begin{circuitikz}
            \tikzstyle{every node}=[font=\large]
            \draw [ fill={rgb,255:red,195; green,232; blue,235} ] (-1.25,17.25) rectangle  node {\large Battery 1} (3.75,14.75);
            \draw [ fill={rgb,255:red,195; green,232; blue,235} ] (-1.25,11) rectangle  node {\large Battery 2} (3.75,8.5);
            \draw [ fill={rgb,255:red,218; green,187; blue,217} ] (-1.25,4.75) rectangle  node {\large USB} (3.75,2.25);
            \draw [ fill={rgb,255:red,233; green,235; blue,199} ] (8.75,17.25) rectangle  node {\normalsize 5V buck (2A)} (13.75,14.75);
            \draw [ fill={rgb,255:red,233; green,235; blue,199} ] (8.75,11) rectangle  node {\large 3.3V Buck (200 mA)} (13.75,8.5);
            \draw [ fill={rgb,255:red,236; green,195; blue,195} ] (18.75,17.25) rectangle  node {\large Servo and power circuits} (23.75,14.75);
            \draw [ fill={rgb,255:red,158; green,148; blue,213} ] (18.75,11) rectangle  node {\large Logic IC and analog circuits} (23.75,8.5);
            \draw (6.25,4.75) to[D] (6.25,8.5);
            \draw [short] (3.75,3.5) -- (6.25,3.5);
            \draw [short] (6.25,4.75) -- (6.25,3.5);
            \draw [short] (3.75,9.75) -- (8.75,9.75);
            \draw [short] (6.25,8.5) -- (6.25,9.75);
            \draw [short] (3.75,16) -- (8.75,16);
            \draw [short] (13.75,16) -- (18.75,16);
            \draw [short] (13.75,9.75) -- (18.75,9.75);
            \draw [dashed] (6.25,9.75) -- (6.25,16);
            \node [font=\large] at (5,16.5) {7.4V - 8.2V};
            \node [font=\large] at (5,10.25) {4.3V - 8.2V};
            \node [font=\large] at (15.25,16.5) {5V};
            \node [font=\large] at (15.25,10.25) {3.3V};
            \draw [dashed] (6.25,16) -- (6.25,18.5);
            \draw [dashed] (6.25,18.5) -- (16.25,18.5);
            \draw [dashed] (16.25,18.5) -- (16.25,16);
        \end{circuitikz}
    }%
    \label{fig:power_tree}
    \caption{}
\end{figure}
\FloatBarrier

On this schematic, there two main regulators, that are, in fact buck switching regulators.
But, there's three power sources :
\begin{itemize}[noitemsep]
    \item   Vbus : The 5V that came from the USBC port when the board is plugged on a PC.
    \item   Vbatt : A battery that is charged to power up the MCU and all of the logic
          circuits.
    \item   Vbatt\_pyro : A battery that is charged to power up the servo engines and
          the thruster starter.
\end{itemize}.

As we can see in \ref{fig:power_tree}, we can configure the power supply as we need.
From the three sources, we're able to use a single one by tying both supply together.
And, we can even then supply the whole system with a single USB supply, but at a reduced
voltage. At the opposite, if needed, we can bypass the 5V buck system if we want to
power the servos and the engines with an higher voltage.

Selection is mainly done by jumper, which are big zero ohm resistors.

\subsection{DCDC buck design}
To design theses circuits, we used a reliable tool that may be found online, from
Texas Instruments : \cite{POWERDESIGNER}. We pass them the input conditions, such as
voltage, and the output wanted : voltage and current. The tool output then circuits
than can match the needs, and we need to select one, based on some settings :

\begin{itemize}[noitemsep]
    \item   Space on board
    \item   Cost
    \item   switching frequency
\end{itemize}

Since we didn't require specific criterion on theses aspects, we choose the one that
was the easiest to solder. Then, we import the designed module into the schematic.
