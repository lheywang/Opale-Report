\section{Power section}
Now, let's look into the power section of the board. This is this section
responsible to power up the heaviest devices, which can't be powered from a
GPIO.

There's few devices on the board that require specific circuits to be driven :

\begin{itemize}[noitemsep]
    \item   The servomotors (ailerons and parachute).
    \item   The buzzer.
    \item   The engine starters.
\end{itemize}

Each of these elements go it's solution since they're requirements aren't the
same.

\subsection{Servomotors}
These devices are already an integrated solution, we don't directly drive the
servo. So, their power requirements are quite easy to fulfill : We feed them
with 5V, stable voltage.

We only used bulk capacitors near their connectors to ensure that a current
peak is handled it locally.

\subsection{Buzzer}
This second device is a different beast, since we drive it directly. This time,
we placed it directly on the supply, with an NMOS on the low side, to switch on
or off the current on it, and thus make sound, or not.

So, we needed an NMOS transistor to be choosen that can handle some current,
but more importantly : An NMOS that was compatible with our voltage level
available on the output of GPIOs\footnote{ The mosfet we choosed was working,
    the can be optimized : We didn't make sure that is was fully switched at our
    GPIO voltage level, and thats effectively the case : It fully switch at arround
    6V. Thus, the current is limited, which is fine for logic usages but not power
    ones. }.

\subsection{Engine starters}
For these last one, this is where our requirements are the highest : We need a
lot of current, around 2A per engine to start, at a quite high voltage (8-9V).

The power supply for this section is different than the remaining of the board,
for a simple reason : When the engine start, the supply is near the short
circuit. It's voltage will then drop, and it may place others components in
UVLO (UnderVoltage LockOut), where the device stop working.

Here we used ULN2803C drivers IC, made to handle a lot of current with
difficults loads (inductives, high voltage...). These are Darlington BJT
transistors pairs, with each channel done for 500mA. By using four channels in
parallel, we got our current requirements.

But, these IC require arround $ 1 \si{\milli\ampere}$ per input ! Here, that
wasn't an issue since these inputs are controlled by an AND gate
\ref{subsec:dis_logic}. This gate is able to provide more than enough current,
while preserving a low input current, perfect for an MCU or any GPIO.
