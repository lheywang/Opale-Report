\section{Parachute mechanism}
Once launched, the rocket need an safe method to land back on earth. This mean
we need a parachute to be deployed once the rocket has got it's maximal altitude.

But, this parachute need to be safely stored inside on rocket tube while launching,
to prevent any damages !

\subsection{Constraints}
To ensure a correct flight trajectory, we needed to place the parachute (which is a quite
heavy element) near the center of the rocket. This mean there is some stuff above it.

This is an serious constraint, because we can't simply eject the tip of the rocket to
free the parachute, but we need to split the tube in half, while ensuring structural
rigidity of the tube while launching !

\subsection{Solutions}
To circumvent theses constraints, we choose a locking mechanism, based on a rotational lock.

There is two rings, one fixed to the tube, the other fixed to an axle in the middle of the
rocket. Theses two are bound together by four locks, for which a rotating movement of the base
will make them rotate, exposing a solid arm to the outside.

Theses arms are going into slots in the second part of the tube, where they lock themselves into,
maintaining the both part of the tube together.

A final spring has been added in the inferior part of the tube to ensure the top part will be
ejected once the locks are removed.