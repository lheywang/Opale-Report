\section{Device drivers}
Next, let's ramp a bit higher in the software structure. Now that we have drivers for
most simple peripherals, we'll need drivers for devices, to handle all of the specific
protocols for us.

Thus, when we need to read a temperature, we don't want to write to a register, wait for
some time, and then read. We want to call a function that return us the temperature.

That's the exact job for a device-driver.
It format requests, use the peripherals drivers to handle the IO, and then make some
calculations to return us a precise value in any cases.

These kind of drivers can be sourced from the manufacturer, but some need to be wrote by
hand. Others need to be modified to be compatible with our software decisions. Some
other only provide standard libraries for the drivers, and let the user develop they're
own drivers.

\subsection{Manufacturer sourced drivers}
In the first case, which is the preferred option, the work needed is generally small.
For example, the driver for the IMU was entirely wrote by Bosch, which require us to
write only the communication functions !

This look like :
\inputminted[linenos, firstline=19471, lastline=19515]{cpp}{\Code/drivers/bno055/bno055.cpp}

And, that's done for the driver !

\subsection{Libraries based drivers}
In this second case, the manufacturer only provided some standard libraries to be used.
An lot of work was needed to ensure this driver is correctly working.

In the previous code section, the manufacturer provided a write and read functions to be
called. It only required us to fill them.

\inputminted[linenos, firstline=292, lastline=330]{cpp}{\Code/drivers/teseo/teseo.cpp}

Hopefully, manufacturer provided us some functions to check if a command was well
formed, to match the checksum, as well as some parsers to identify the different
elements.

This save us a lot of time compared to writting the full driver.

In this example of code, the manufacturer provided the GNSS\_PARSER\_CheckSanity functions
and other return values codes.

We only needed to implement the IO.

\subsection{Hand written drivers}
Even if theses drivers are quite complex to write, they're generally reserved to much
smaller chips, which make the develop quite straightforward.
For example, the only driver we needed to develop like that was for the temperature
sensor.

\inputminted[linenos, firstline=119, lastline=158]{cpp}{\Code/drivers/ms5611/MS5611.cpp}

This driver include all of the calculations needed for temperature corrections
needed to match the precision. Theses are indeed affected by temperature,
or just measure range. The sensor isn't linear at all over the plage, but per
segments.

Here, all of the code was hand written because nothing, except documentation was provided.


\subsection{Final words}
To conclude this part on drivers, we can resume them as a fundamental brick of the software,
that handle all of the device specific requirements.
This part part is generally where the errors are, because they're difficult to test entirely.