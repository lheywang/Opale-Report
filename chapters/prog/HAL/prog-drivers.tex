\section{Drivers}
Next in the hierarchy, we found the drivers. This codes are used to prevent from the
user to perform low-level IO, which are error prones.

\paragraph{}
All of them are written by the manufacturer or the chip, in our case, Nordic
Semiconductors.
Theses contain mostly definitions about structures passed to the driver to 
configure an element, of function that we may call.

\subsection{Calling the driver}
Theses drivers can be called by two distinct ways : 
\begin{itemize}
    \item From bare-metal code
    \item From Zephyr RTOS drivers, who redirect calls to the bare-metal code
\end{itemize}

The second method is the safest, in the manner that the RTOS handle the request for us, 
but, to ensure compatibility with a lot of chips, it can't provide exact same options as
the bare metal driver.

That's why we used both, mostly the Zephyr one were we could.

\subsection{Driver usage}
\subsubsection{Bare metal}
To give an example of the driver beeing used, there is the initialization code the ADC module 

\inputminted[linenos, firstline=103, lastline=112]{cpp}{\Code/peripherals/saadc/saadc.cpp}

This code is responsible for the configuration of some advanced behaviors of the ADC peripheral,
such as setting it's resolution, callback function and so.
Calls are here quite long, because of the different structs that need to be configured, defined,
and applied for each aspect.

\subsubsection{Zephyr}

On the other hand, here the example for the Zephyr driver being called.
This is much simpler, because we're here calling the Zephyr Driver that handle a lot of the work 
for us, based on the devicetree !

\inputminted[linenos, firstline=100, lastline=102]{cpp}{\Code/devices/servo/servo.cpp}

We just need to care about the pulse length we want to set. All the other parameters were defined
when the kernel was booting.






