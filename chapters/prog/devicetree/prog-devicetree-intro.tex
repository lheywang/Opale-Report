\section{Introduction to devicetrees}
First, we need to explain what is a devicetree, because that's actually far from begin clear and easy to understand.
The wikipedia article describe it concisely, here a quotation :

\begin{quote}
    \quote{In computing, a devicetree (also written device tree) 
    is a data structure describing the hardware components of 
    a particular computer so that the operating system's kernel 
    can use and manage those components, including the CPU or CPUs, 
    the memory, the buses and the integrated peripherals.}
    \cite{devicetree}
\end{quote}

\paragraph{}
So, we know that we're going to found some hardware description in theses files. Theses kind of files are 
commonly used on the Linux kernel for this purpose, on hardware that is much faster than our small 
microcontroller. 
This way of describing hardware, even if it's quite difficult to get it rigth enable some extremely smart
options, such as dynamic reconfiguration.

\paragraph{}
In fact, for other controllers we're supposed to bind pins by hand by correctly assigning register values.
This is easy to develop, but once you want to change something, it become difficult to not make mistake by 
misreading a value.
This problem is solved with devicetrees, since they store hardware config for ourselves. And, they can 
even store multiple configuration, and offer the option to switch at compile time.

It's then possible to develop a devicetree for the development board, and for the final board, and within
one parameter change between them !

\paragraph{}
Devicetrees are written in plain text, and there can be only a single file used for the compilation at 
a given time. Theses files use the extension ".dts". This main file contain the root, also named "/", 
as any UNIX filesystem. Then, we add "folders", which are named nodes to it. Nodes can be 
nested inside others to make the code cleaner. Theses are sometimes called sub-nodes. And inside nodes, 
there is some properties, that can be seen as a variable that configure one aspect\footnote{
    Properties can be accessed by the software, thus they may not describe an hardware caracteristic but
    some boundaries imposed by the hardware to the software. This can substitue to \#define macros in plain
    C.
}.

As an example, a node may be RAM, and properties are the size, the speed and any other hardware
parameters.

In correctly designed devicetrees, there shall be one node for each device that can't be detected on 
runtime. This include I2C slaves.

\paragraph{}
Each node can get a "compatible" property. This enable a very usefull tool on the compilation step to ensure
our node is well formed. This property is simply a list of required properties, and they're size. That's a 
usefull verification tool, because device tree compiler \textit{can't} check for errors. If leaves you in a 
state with invalid devicetree files, and near nothing to debug it.

\paragraph{}
We can easily image that theses files will become big, even for simple systems. To give an idea, there is
arround one thousand of lines just for our simple microcontroller !

Thus, the developpers of Device Tree Compilers managed to create an include syntax, based on overlays files
(".dts\textbf{i}"). Theses are included by after and contain only the code for a single peripheral.
Then, we include them over the main devicetree file. 

If the new nodes were not present, they're added to the main code. If they were already existing, the new 
file will overhide properties. 
It's always the latest added that take the priority.



