\section{RTOS}
For this second part, we'll go a bit higher in the software structure of the project, and see the different
layers. The first layer is the one we'll describe here, which is the RTOS.

This stand for Real Time Operating System.

\begin{quote}
    \quote{A real-time operating system (RTOS) is an operating system (OS) for real-time
        computing applications that processes data and events that have critically defined
        time constraints. A RTOS is distinct from a time-sharing operating system, such as
        Unix, which manages the sharing of system resources with a scheduler, data buffers,
        or fixed task prioritization in multitasking or multiprogramming environments. All
        operations must verifiably complete within given time and resource constraints or
        else fail safe. Real-time operating systems are event-driven and preemptive, meaning
        the OS can monitor the relevant priority of competing tasks, and make changes to
        the task priority.}
    \cite{RTOS}
\end{quote}

\paragraph{}
Thus, we're able to specify strict time constraints for different threads. This is very usefull, since it
enable us to ensure the controller, designed for a specific sampling rate will iterate at a defined speed.

\subsection{Basic RTOS concepts}
To match they're requirements, near all of the RTOS define some basic concepts, such as :

\begin{itemize}
    \item A scheduler (that may be able to preempt a task)
    \item Some communication protocols :
          \begin{itemize}
              \item Queues
              \item Messages
              \item events
          \end{itemize}
    \item Tasks (which include a priority flag !)
\end{itemize}

\subsubsection{Scheduler}
This is the main aspect of the RTOS, because this the task responsible to schedule other tasks,
while ensuring real times constraints. There's a lot of different architecture here, each adding
it's own set of positives or negatives aspects.

\subsubsection{Communication protocols}
For this second concepts, we need to present the differents ways to send data from one task to the
other.

Since we're running multiple task in parallel, we can't define which instruction will be executed
before another on another task. Thus, standard memory transfers, based on variables and pointers
become unreliable.
In fact, most of the RTOS block theses kind of transfers !\footnote{
    The name may be different from one RTOS to the other, but the concepts remains the same.
    Here, we used the Zephyr RTOS naming convension, as we used this RTOS by after.
}

Thus, we're forced to use thread-safe memory transfers, which are based on buffers, FIFOs and other
principle. Since the RTOS manage theses, it can enable, or disable some operations on the shared buffer.
This is done to ensure memory safety during the execution of the program.

There's indeed multiple type of transfers, depending the needs.
The first, Queues are FIFOs, and able to transfers large amount of data while ensuring the order. The
drawback of them is they're memory footprint which can be quite large.

The second ones are similar, but only enable a small amount of data to be transfered. And, as opposed
to the FIFOs, when we overwrite the data, it's deleted rather than pushed on the Queue. It's memory
footprint is similar as the size of a single object, but, it won't enable the conservations of previous
sample. Thus, we may skip data.

The last one are Events, which are simply bit. We flip them, to notify a thread that something can, or
can't be done.

\subsubsection{Tasks}
The last concept is the Task. This correspond to a thread that is going to be executed on the CPU.
For example, a task may be the main as we know.

Since tasks are handled by the RTOS, it's possible to run multiple tasks in parallel, making control
logic simpler. Thus, the timing logic (Timestamps...) can be removed, and replaced by a delay. The RTOS
will then pause the task and execute another for the required duration.