\section{Embedding our results in the software}

Now that we've seen how to simulate our code and the global architecture, we
will now see how we integrate this to command our control surfaces. The thing
is that because we've built our simulation first, we just need to code in C
the same way we did our simulation. So basically, we just input this code:

\inputminted[linenos, firstline=1, lastline=29]{c}{\RawCode/code.c}

Since we already have different gains, we can just input those into a PID
and give the desired angle. Since we want the rocket to go upright and that
the IMU is positioned in such a way that when the rocket is upright, we have
a $0 \si{\degree}$ angle. We also need the rocket not rotating on itself, and
we can tell the controller everyone of those by inputting in the $PID_X$,
the $PID_Y$ and the $PID_Z$ a $0 \si{\degree}$ desired angle.

The controller also has other uses such as controlling the gate to start the
motor. This part is linked to the security part of the software which only
activate this part when everything is good to go and that the rocket has our
permission to fire.
Another use is parachute deployment. For this part, we need to detect that
we have had a liftoff, that we are at the apogee, which we can check by
checking the vertical acceleration, inversing and with the altitude starting
to decrease. To protect ourselves against false positives we can just check
if we are at least 10m above the starting altitude of the rocket.
