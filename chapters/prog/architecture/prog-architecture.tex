\section{Architecture}
For this final section on the software architecture, let's talk about the top of the pyramid.
Now, we have a fully configured chip, devices drivers, and an RTOS ready to handle task.

We just need to create the differents elements to communicate, and connect them together.

Theses elements are :

\begin{itemize}[noitemsep]
    \item   A safety task, charged to ensure the authorisation to start.
    \item   A measurement task, charged to acquire all of the sensor data on the rocket.
    \item   A logger task, charged to log events and measure for a further usage.
    \item   A controller task, charged to compute and apply the commands for the different actuators.
\end{itemize}

Each of theses task communicate which each other trough three ways :

\begin{itemize}[noitemsep]
    \item   Some FIFO, from one task to the other. Used for data exchanges such as measurement data mostly.
    \item   A global status register, where each bits has it's own meaning.
    \item   A global messaging systems, where each thread can sen a global message to each others. Used to indicate errors codes.
\end{itemize}

Each of theses tasks has some input data, that came from the measurement task, and some outputs that are
reserved peripherals or part of peripherals. The software is done to ensure a single task will access to the SPI peripheral.

Graphically, this may look like :

\begin{figure}[!hbt]
    \centering
    \resizebox{\SchematicWidth}{!}{%
        \begin{circuitikz}
            \tikzstyle{every node}=[font=\normalsize]
            \draw [ fill={rgb,255:red,158; green,148; blue,213} ] (1.25,8.5) rectangle  node {\normalsize Measurement task} (7.5,3.5);
            \draw [ fill={rgb,255:red,236; green,195; blue,195} ] (11.25,14.75) rectangle  node {\normalsize Safety task} (17.5,12.25);
            \draw [ fill={rgb,255:red,236; green,195; blue,195} ] (11.25,7.25) rectangle  node {\normalsize Controller task} (17.5,4.75);
            \draw [ fill={rgb,255:red,236; green,195; blue,195} ] (11.25,-0.25) rectangle  node {\normalsize Logger task} (17.5,-2.75);
            \draw [->, >=Stealth] (7.5,6) -- (11.25,6);
            \draw [->, >=Stealth] (10,-1.5) -- (11.25,-1.5);
            \draw [->, >=Stealth] (10,13.5) -- (11.25,13.5);
            \draw [short] (10,-1.5) -- (10,4.75);
            \draw [short] (7.5,4.75) -- (10,4.75);
            \draw [short] (10,13.5) -- (10,7.25);
            \draw [short] (7.5,7.25) -- (10,7.25);
            \draw [ fill={rgb,255:red,162; green,199; blue,179} ] (20,-0.25) rectangle  node {\normalsize SPI EEPROM} (26.25,-2.75);
            \draw [ fill={rgb,255:red,162; green,199; blue,179} ] (20,11) rectangle  node {\normalsize Wings control} (26.25,8.5);
            \draw [ fill={rgb,255:red,162; green,199; blue,179} ] (20,14.75) rectangle  node {\normalsize Status LED} (26.25,12.25);
            \draw [ fill={rgb,255:red,162; green,199; blue,179} ] (20,7.25) rectangle  node {\normalsize Wings control} (26.25,4.75);
            \draw [->, >=Stealth] (17.5,6) -- (20,6);
            \draw [->, >=Stealth] (17.5,13.5) -- (20,13.5);
            \draw [->, >=Stealth] (17.5,-1.5) -- (20,-1.5);
            \draw [->, >=Stealth] (18.75,9.25) -- (20,9.25);
            \draw [->, >=Stealth] (18.75,10.25) -- (20,10.25);
            \draw [short] (18.75,9.25) -- (18.75,6)node[pos=0.5, fill=white]{Command};
            \draw [short] (18.75,10.25) -- (18.75,13.5)node[pos=0.5, fill=white]{Enable};
        \end{circuitikz}
    }%
    \caption{Task organizations}\label{tkz:tasks}
\end{figure}
\FloatBarrier

This architecture has been choose to ensure a sepration between tasks, and thus some security. If something went wrong on the logger
task, the controller would not be affected, because they're independant.

