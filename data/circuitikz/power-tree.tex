\begin{figure}[!ht]
    \centering
    \resizebox{1\textwidth}{!}{%
        \begin{circuitikz}
            \tikzstyle{every node}=[font=\normalsize]
            \draw (3.75,9.75) to[short, -o] (2.5,9.75) node[left] {Vbatt (pyro)};
            \draw (3.75,14.75) to[battery ] (3.75,12.25);
            \draw (3.75,9.75) to[battery ] (3.75,7.25);
            \draw (3.75,19.75) to[battery ] (3.75,17.25);
            \draw (6.25,19.75) to[D] (8.75,19.75);
            \draw (3.75,14.75) to[short] (10,14.75);
            \draw (3.75,19.75) to[short] (6.25,19.75);
            \draw (8.75,19.75) to[short] (10,19.75);
            \draw (10,19.75) to[short] (10,14.75);
            \draw (3.75,17.25) to (3.75,16.5) node[ground]{};
            \draw (3.75,12.25) to (3.75,11.5) node[ground]{};
            \draw (3.75,7.25) to (3.75,6.5) node[ground]{};
            % Start Dual inline chip
            \foreach \x in {0,...,1}{
            \pgfmathtruncatemacro\y{\x + 1}
            \pgfmathtruncatemacro\z{\x + 3}
            \draw  (12,19.75-\x/2) -- ++(0.5,0);
            \draw  (14.25,19.75-\x/2) -- ++(0.5,0);
            \node [font=\tiny, align=left] at (12.75, 19.75-\x/2) {\y};
            \node [font=\tiny, align=left] at (14, 19.75-\x/2) {\z};
            }
            \draw  (12.5,20.25-\x/2) rectangle   (14.25,18.75-\x/2);
            \draw  (13.625,20.25-\x/2) arc [start angle=0, end angle=-180, radius=0.25cm];% End Dual inline chip
            
            % Start Dual inline chip
            % End Dual inline chip
            
            % Start Dual inline chip
            \foreach \x in {0,...,1}{
            \pgfmathtruncatemacro\y{\x + 1}
            \pgfmathtruncatemacro\z{\x + 3}
            \draw  (12,9.75-\x/2) -- ++(0.5,0);
            \draw  (14.25,9.75-\x/2) -- ++(0.5,0);
            \node [font=\tiny, align=left] at (12.75, 9.75-\x/2) {\y};
            \node [font=\tiny, align=left] at (14, 9.75-\x/2) {\z};
            \draw  (12.5,10.25-\x/2) rectangle   (14.25,8.75-\x/2);
            \draw  (13.625,10.25-\x/2) arc [start angle=0, end angle=-180, radius=0.25cm];% End Dual inline chip
            }
            
            \draw (12,19.75) to[short] (10,19.75);
            \draw (15,19.75) to[short, -o] (16.25,19.75) node[right] {MCU supply};
            \node at (10,19.75) [circ] {};
            \draw (12,9.75) to[short] (3.75,9.75);
            \draw (15,9.75) to[short, -o] (16.25,9.75) node[right] {Servo supply};
            \draw [dashed] (10,14.75) -- (10,9.75);
            \draw [dashed] (10,12.5) -- (15,12.5);
            \draw [dashed] (15,12.5) -- (15,9.75);
            \draw (12,9.25) to (12,7.75) node[ground]{};
            \draw (15,9.25) to (15,7.75) node[ground]{};
            \draw (12,19.25) to (12,18) node[ground]{};
            \draw (15,19.25) to (15,18) node[ground]{};
            \draw (3.75,19.75) to[short, -o] (2.5,19.75) node[left] {Vbus (5V)};
            \draw (3.75,14.75) to[short, -o] (2.5,14.75) node[left] {Vbatt};
        \end{circuitikz}
    }  
    \label{fig:power-tree}
\end{figure}
\FloatBarrier