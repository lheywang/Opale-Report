\paragraph{}
This is the point at which we were able to achieve this in approximately
eight months. The project is extremely dense, encompassing a wide range
of disciplines, including electronics design, flow mechanics, conventional
mechanics and even more.

\paragraph{}
This project was an eye-opening experience considering the complexity
and the need to comprehend so many different phenomena.
To sum up our work order in terms of parts, we first designed our rocket
in a gross simulation, we then designed in parallel the rocket in 3D and
did the electronics part. This last part was very time consuming to choose
every component and to build and encode the ecosystem of peripherals around
our MCU.

\paragraph{}
Once our first 3D rocket design was finished, we defined our constants like
the inertia matrix needed for our simulation. Once this was finished, we
developed our simulation using Simulink and a lot of help from Mr Schwartz,
Mr.Laroche and Mr.Nageotte who guided us and showed us what work we could
take inspiration from because this simulation isn't entirely self-made.
Every equation and every link between real-life and numerical equation was
made thanks to the work of those cited in the bibliography.

\paragraph{}
We are now at this point where we calculated and simulated our rocket to
start making it and where we have a mostly working PCB. We now need to print
and assemble our rocket and finish coding our MCU.
To sum up, this project was a "multidisciplinary nightmare" to call it badly
because we had to learn a lot of things which interacted between each other
as well as how to simulate it accurately to determine many parameters of
our rocket. The theory is done, we just need to finish the practice.

\paragraph{}
We would also like to thank all the teachers and educational staff for
their invaluable help throughout the project.

\vspace{5cm}
\begin{minipage}[c]{1\textwidth}
    \raggedleft
    We would like to express our gratitude to all those who have taken the time to read this document. \\
    BRANDSTAEDT Arthur \\
    HEYWANG Léonard
\end{minipage}
