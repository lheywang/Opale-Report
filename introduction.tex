\paragraph{}
This document cover the Opale project, which took place in the context inside the Master SEME
on the univeristy of Strasbourg.

\paragraph{}
This project was started on the first semester, by november 2024. The goal was to develop a
small scale rocket that modelize the behavior of real ones. This model include the usage
of powder engines, as well as come control electronics embedded into the rocket.

\paragraph{}
For all along this report, we're going to describe all the reflexions, studies, and conception.
Since we're Electronics Engineering students, we'll focus more on the electronics and sofware aspects,
but the mechanical part will also quicly be explained.
The goal with the control electronics is to ensure the rocket will follow a defined trajectory, as
well as storing into an EEPROM some measurements for an after fly interpretation.

The rocket can be modeled as a complex system, composed with multiple blocks :

\begin{itemize}[noitemsep]
    \item   Sensors (positionning, temperature...)
    \item   Outputs (Servos engines control, engine starter...)
    \item   Controller (MCU + Memory)
    \item   Power supplies
\end{itemize}

\paragraph{}
This report will be splitted in multiple parts, that are not ordered in a chronological order, but
more in a logic way.

\paragraph{}
In the first chapter, we're going to explain how we choosed main hardware elements, such as the controller,
the servo engines or the inertial measurement unit. Theses are device where a wide range of choices are possible !

In a second part, we'll focus on the caracterisation of sensors and actuators, to understand precisely how they behave to
any input signals. This is an important step if we want to develop a precise controller for them !

For a third chapter, we're going to detail the whole procedure of the electronic design. This start from the schematic,
and end up with a whole printed circuit board, ready to be manufactured.

For the next part, we'll see how we designed the controller of the rocket, from the physical equations to a simulation,
and then implemented into an embedded controller.

And, to finish the main section, a final part will be dedicated for the whole software we wrote, that handle the control as
well as some other task on the rocket.

After the conclusion, in annexes there will be some more details. This include a mechanical overview of the rocket, as well as
the process of soldering the board or some precisions about the software and controller we developped.

