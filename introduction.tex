\paragraph{}
This document cover the Opale project, which took place in the context inside the Master SEME
on the univeristy of Strasbourg.

\paragraph{}
This project was started on the first semester, by november 2024. The goal was to develop a 
small scale rocket that modelize the behavior of real ones. This model include the usage 
of powder engines, as well as come control electronics embedded into the rocket.

\paragraph{}
For all along this report, we're going to describe all the reflexions, studies, and conception.
Since we're Electronics Engineering students, we'll focus more on the electronics and sofware aspects, 
but the mechanical part will also quicly be explained.
The goal with the control electronics is to ensure the rocket will follow a defined trajectory, as 
well as storing into an EEPROM some measurements for an after fly interpretation.

The rocket can be modelled as a complex system, composed with multiples blocks : 

\begin{itemize}
    \item   Sensors (positionning, temperature...)
    \item   Outputs (Servos engines control, engine starter...)
    \item   Controller (MCU + Memory)
    \item   Power supplies
\end{itemize}

\paragraph{}
This report will be splitted in multiples parts, that are not ordered in a chronological order, but 
more in a logic way.

\paragraph{}
In the first time, we're going to look at the measures we done on the different sensors and actuators, to understand
how they respond to our inputs, what points are importants, and so.
A mechanical part will follow this first part, we're we describe some mechanical structure of the rocket. 
In a third part, we're going to look at the control theory we've done on the rocket, to design the 
controller algorithms. This is extensively done on simulation.
Next, we're going to dig into the electronic design, which include the design of an printed
circuit board with all of our sensors, actuators, and so.
Rigth after, a part with all of the programming steps will show how the MCU is configured, and how we structured the 
controller to ensure a precise response, as well as some security functions.